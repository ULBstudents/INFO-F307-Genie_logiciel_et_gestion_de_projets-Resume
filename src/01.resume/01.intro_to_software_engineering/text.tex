\section{Introduction au Software Engineering}



\subsection{Qu'est-ce que le Software Engineering ?}
\textit{"L'état de l'art de développer un Software de qualité à temps et dans un certain budget."}
\\Le Software Engineering est
\begin{itemize}
    \item le fait de développer des logiciels en tirant le meilleur parti entre la perfection et les contraintes physiques ;
	\item le travail d'équipe avec une bonne communication.
\end{itemize}




\subsection{Pourquoi le Software Engineering est très important ?}
Le Software Engineering évite
\begin{itemize}
    \item les mauvais fonctionnements du Software ;
	\item des conséquences excessif.
\end{itemize}
Par exemple la fusée Ariane 5 qui a explosé suite à un petit bug dans un programme.
\\C'est encore aujourd'hui le bug informatique le plus chère de l'histoire.

\subsection{Pourquoi un Software échoue ?}
Les Softwares échouent pour une ou plusieurs raisons :
\begin{itemize}
    \item Hors budget ;
    \item Retard, le client n'en aura peut être plus besoin ;
    \item Ne correspond pas aux exigences du client ;
    \item Qualité inférieure qu'initialement prévu ;
    \item Performances ne répondent pas aux attentes ;
    \item Trop difficile à utiliser.
\end{itemize}



\subsection{Les quatre P}
\begin{description}
    \item [People] Il y a plein de groupes de personnes qui interviennent sur le projet.
		\begin{description}
			\item [Business management] est responsable du budget encluant le profil, les couts effectifs ainsi que la satisfaction du client.
			\item [Project management] est responsable du panning et du suivi du projet.
			\item [Development team] est responsable du développement et de la maintenance du projet.
			\item [Customers] est responsable de l'achat du Software.
			\item [End users] interagit et utilise le Software après que celui-ci soit fini d'etre développé.
		\end{description}
    \item [Product] Le code, le produit compilé, la doc, les tests, ...
		\begin{description}
			\item [Documentation] sont les documents produits pendant la définition et le développement du Software.
			\item [Code] sont les sources.
			\item [Customer documents] sont les documents expliquant comment utiliser le produit.
			\item [Productivity measurements] analyse la productivité du projet.
		\end{description}
    \item [Project] Défini les activités et les résultats attendu associés à la production du Software.
		\begin{description}
			\item [Planning] est le plan, moniteur et controle du Software.
			\item [Requirements analysis] défini se qui doit etre construit.
			\item [Design] décrit comment construire le Software.
			\item [Implementation] Programme le Software.
			\item [Testing] Valide le Software avec tous ses besoins.
		\end{description}
    \item [Process] Manière d'organiser les gens, la discipline, la structure.
		\begin{description}
			\item [Guide] impose une structure et aide les gens à se dirigier sur les bonnes activités.
			\item [Phase] impose une inter-relation, en définissant l'ordre et la fréquence jusqu'à la livraison du projet.
		\end{description}
\end{description}



\subsection{Principes du Software Engineering}
Les plus importants :
\begin{itemize}
   \item La qualité est de première importance.
   \item Un logiciel de haute qualité est possible.
   \item Donner le produit aux clients le plus rapidement possible pour avoir leurs avis.
   \item Utiliser un processus de développement adapté.
\end{itemize}



\subsection{Éthiques du Software Engineering}
Il y a une part d'éthique. On ne copie/colle pas des bouts de codes sur des contrats différents, éthiques entre collègues, etc ...
