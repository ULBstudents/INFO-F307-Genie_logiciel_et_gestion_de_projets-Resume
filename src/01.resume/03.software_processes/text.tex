\section{Software Processes}



\subsection{Qu'est-ce qu'un Software Processes ?}
Un Software est composé d'activités (Planning, design, test, ...) et les activités sont organisés en différentes phases.
\begin{itemize}
    \item Il définit l'ordre et fréquence des phases ;
    \item Il spécifie les critères pour passer d'une phase à l'autre ;
    \item Il définit si un projet est livrable ou non.
\end{itemize}



\subsection{Phases d'un Software Processes}
\begin{description}
    \item [Inception (Lancement)] Le produit logiciel y est conçu et défini.
    \item [Planning (Organisation)] Initialise l'emploi du temps, les ressources et le coût déterminé.
    \item [Requirements Analysis (Besoin d'analyse)] Structure et spécifie les besoins ("Quoi").
    \item [Design (Conception)] Structure et spécifie une solution ("Comment").
    \item [Implementation (Mise en oeuvre)] Construit une solution du logiciel.
    \item [Testing (Test)] Valide la solution en fonction des besoins.
    \item [Maintenance (Entretien)] Répare les défauts et adapte la solution aux nouveaux besoins.
\end{description}



\subsection{Modèles de Software Processes}
\subsubsection{Waterfall}
Le processus classique utilisé comme modèle est le développement logiciel étape-par-étape \textbf{waterfall}.
\\On analyse les besoins du client, on design l'application, on la code, on la lire et ensuite on la maintient.
\\On termine une étape avant de commencer la suivante.

C'est facile à mettre en place mais on ne reçoit aucun retour du client. Donc si il y a une erreur dans la compréhension des besoins du clients ; on ne s'en rend compte que à la fin du processus et il est trop tard. On doit alors revenir à la première étape et tout refaire, le processus devient donc long et peut être coûteux.



\subsubsection{Itératif et incrémental}
Le développement \textbf{itératif} est un waterfall avec un feedback du client entre chaque étape.
\\Le développement \textbf{incrémental} consiste à développer le produit fonctionnalité par fonctionnalité en livrant à chaque fois une version préliminaire du projet (Processus composé de mini-waterfall).
\\La \textbf{livraison incrémentale} est la même chose mais appliqué à la livraison.



\subsubsection{Prototyping}
Prototyping est le processus de création d'un modèle incomplet du futur programme logiciel qui peut être utilisé pour des tests, exploration ou pour valider une hypothèse.


\subsubsection{Unified Process}



\subsubsection{Open Source}
