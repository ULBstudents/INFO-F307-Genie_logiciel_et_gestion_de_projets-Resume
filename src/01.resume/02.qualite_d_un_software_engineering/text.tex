\section{Qualité d'un Software Engineering}



\subsection{Les défauts dans un Software}
Un Software peut contenir des défauts, ces défauts sont des dérives des besoins.
\\La qualité du Software concerne la sévérité et la grandeur de ces défauts.
\\Pour obtenir une bonne qualité, il faut :
\begin{itemize}
	\item En retirer un maximum avant que le projet ne soit fini ;
	\item En retirer un maximum dès le début du développement.
\end{itemize}


\subsection{Le cout des corrections des défauts}
Chacune de ces étapes coutent de l'argent et au final on arrive à une grande somme.
\\Requirements (\$1) $\Rightarrow$ Design (\$5) $\Rightarrow$ Implementation (\$20) $\Rightarrow$ Test (\$50) $\Rightarrow$ Maintenance (\$100)



\subsection{Vérification et validation}
\begin{description}
    \item [Vérification] S'assurer à ce que chaque module est construit conformément à ses spécifications. "Est-ce qu'on construit le produit correctement ?". Surtout inspections et révisions.
    \item [Validation] Vérifier que chaque module terminé satisfait ses spécifications. "Est-ce qu'on construit le bon produit ?". Surtout des tests.
\end{description}



\subsection{Metrics}
\textit{Mesures numérique qui quantifie le degré qu'un Software ou un Process Processes possède comme attribut.}
\\Il est collecté et analysé tout au long du Software Project.
\\Aide à
\begin{itemize}
    \item Déterminer le niveau de qualité d'un Software ;
	\item Estimer le calendrier d'un project ;
	\item Suivre la progression au calendrier ;
	\item Déterminer la taille et la complexité d'un Software ;
	\item Déterminer le prix d'un projet ;
	\item Améliorer le Process.
\end{itemize}



\subsection{Quality Metrics}
Quality Metrics est spécialement concentré sur la caractère "qualité" du Software et des procédés employés pendant le cycle de vie du Software.
\\Cela inclue :
\begin{itemize}
	\item la densité des défauts ;
	\item mean time to failure ;
	\item problème client ;
	\item satisfaction du client.
\end{itemize}