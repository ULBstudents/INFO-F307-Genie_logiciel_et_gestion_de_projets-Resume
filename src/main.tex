\documentclass[a4paper,10pt]{article}
\usepackage[a4paper]{geometry}
\geometry{hscale=0.8,vscale=0.8,centering}
\usepackage[utf8]{inputenc}
\usepackage[T1]{fontenc}
\usepackage[french]{babel} % Exposant


\usepackage{enumerate} % Listes
\usepackage{amsmath} % Matrices
\usepackage{graphicx}
\usepackage{amssymb}
\usepackage{ulem}
\usepackage{color}

\usepackage{listings} % Lecture du code
\usepackage{hyperref} % Hyperlien


% Mise en page spéciale fancyhdr pour les en-têtes
\usepackage{fancyhdr}
\pagestyle{fancy}

\renewcommand{\headrulewidth}{0pt}
\fancyhead[C]{} % Rien en haut de page au milieu
\fancyhead[L]{\leftmark} % Nom du chapitre actuel en haut de page à gauche
\fancyhead[R]{\thepage} % Numéro de page en haut de page à droite

\renewcommand{\footrulewidth}{0pt}
\fancyfoot[C]{} % Rien en bas de page au milieu
\fancyfoot[L]{Open source pour ajout ou modification:\\https://github.com/Rodriguevb/INFO-F307-Genie\_logiciel\_et\_gestion\_de\_projets-Resume/} % Numéro de page en bas de page à gauche
\fancyfoot[R]{\thepage} % Numéro de page en bas de page à droite
\setlength{\headheight}{12.1638pt}


\author{Rodrigue \textsc{Van Brande}} % Auteur
\date{\today} % Date de compilation du pdf

\pdfinfo{
    /Author   (Rodrigue VAN BRANDE)
    /Creator  (https://github.com/Rodriguevb/INFO-F307-Genie_logiciel_et_gestion_de_projets-Resume/)
}

%Création d'un subsub...section (avec \paragraph{} \subparagraph{})
\usepackage{titlesec}

%Profondeur pour la table des matières dans les titres
\setcounter{secnumdepth}{5}
\setcounter{tocdepth}{5}

% On rajoute les espaces
\titleformat{\paragraph}{\normalfont\normalsize\bfseries}{\theparagraph}{1em}{}
\titlespacing*{\paragraph}{0pt}{3.25ex plus 1ex minus .2ex}{1.5ex plus .2ex}
\titleformat{\subparagraph}{\normalfont\normalsize\bfseries}{\thesubparagraph}{1em}{}
\titlespacing*{\subparagraph}{0pt}{3.25ex plus 1ex minus .2ex}{1.5ex plus .2ex}
%\newcommand{\fonction}[nb de parametre]{définition de la commande}
\newcommand{\subsubsubsection}[1]{\paragraph{#1}}
\newcommand{\subsubsubsubsection}[1]{\subparagraph{#1}}


% Titre du PDF
\title{INFO-F307 - Génie logiciel et gestion de projets\\Ragnhild \textsc{Van Der Straeten}\\Résumé du cours}

\pdfinfo{
/Title(INFO-F307 - Génie logiciel et gestion de projets)
}

\begin{document}
    \maketitle       % Titre
    \newpage         % Saut de page
    \tableofcontents % table des matières / Besoin d'une double compilation
    \newpage         % Saut de page

    \subsection{Software Processes}

\subsubsection{Quel est la principale activité du Software Processes ?}
\subsubsection{Quel est le type du principal du Software Processes ?}
\subsubsection{Comment les méthodes agiles sont apparu ?}
\subsubsection{Quel est le principe de l'agilité ?}
\subsubsection{Comment les processus agiles sont effectuées ?}
\subsubsection{Pouvons-nous combiner un processus agile avec un non-agile ?}

    \subsection{Software Processes}

\subsubsection{Quel est la principale activité du Software Processes ?}
\subsubsection{Quel est le type du principal du Software Processes ?}
\subsubsection{Comment les méthodes agiles sont apparu ?}
\subsubsection{Quel est le principe de l'agilité ?}
\subsubsection{Comment les processus agiles sont effectuées ?}
\subsubsection{Pouvons-nous combiner un processus agile avec un non-agile ?}

    
\end{document}